\documentclass[11pt, a4paper]{article}
\usepackage[utf8]{inputenc}
\usepackage[portuguese]{babel}
\usepackage[T1]{fontenc}
\usepackage[margin=2cm]{geometry}
\usepackage{amsmath}
\usepackage{amsfonts}
\usepackage{amsthm}
\usepackage{mathtools}
\usepackage{enumitem}
\usepackage{csquotes}
\usepackage[citestyle=authortitle]{biblatex}
\usepackage[hidelinks]{hyperref}

\addbibresource{bibliografia.bib}

\title{
	Elementos de Análise Infinitesimal I\\
	Actividade Formativa 1\\
	Proposta de Resolução
}

\author{
	Carlos Pinto Machado
	<\href{mailto:2200909@estudante.uab.pt}{2200909@estudante.uab.pt}>
}

\hypersetup{
	pdfsubject = {Elementos de Análise Infinitesimal I},
	pdftitle = {Actividade Formativa 1(Proposta de Resolução)},
	pdfauthor = {Carlos Pinto Machado(2200909)},
	pdfcreator = {},
	pdfproducer = {}
}

\newcommand{\exercicio}[1]{
	\begin{flushleft}
		\large #1.
	\end{flushleft}
	\phantomsection
	\addcontentsline{toc}{section}{Exercício #1}
	\addtocounter{section}{1}
	\setcounter{proposition}{0}
}

\newtheorem{proposition}{Proposição}[section]


\begin{document}
\maketitle
\tableofcontents

\clearpage

\exercicio{1}


\begin{align*}
	\intertext{Começamos por averiguar se a sucessão $\frac{1}{n}$ é
		decrescente, ao calcular a diferença entre um termo $n + 1$ e um termo $n$ e
		verificar se este é inferior a zero.}
	\frac{1}{n + 1}- \frac{1}{n} & = \frac{n - (n + 1)}{n(n+1)} =
	-\frac{1}{n(n+1)}                                                                                                   \\
	\intertext{Dado que $n(n+1) > 0$, para todo o $n \in \mathbb{N}$, podemos
		concluir que a sucessão é decrescente. Deste modo o supremo da sucessão é
		o primeiro termo da sucessão.}
	\text{sup}\left\{\frac{1}{n}: n \in \mathbb{N}\right\}
	                             & = \text{max}\left\{\frac{1}{n}: n \in \mathbb{N}\right\}
	= \frac{1}{1}= 1
	\intertext{O máximo existe dado que 1 pertence ao conjunto. Do mesmo
		modo podemos observar que a sucessão é limitada, dado que para
		qualquer $n \in \mathbb{N}$, $\frac{1}{n} > 0$. Por consequência, o ínfimo
		do conjunto vai ser o limite da sucessão de $n \to \infty$.}
	\text{inf}\left\{\frac{1}{n}: n \in \mathbb{N}\right\}
	                             & = \lim_{n} \frac{1}{n} = 0                                                           \\
	\text{min}\left\{\frac{1}{n}: n \in \mathbb{N}\right\}
	                             & \text{ não existe dado que } 0 \not\in \left\{\frac{1}{n}: n \in \mathbb{N}\right\}.
\end{align*}

\exercicio{2}

\begin{proposition}
	$(n!)^2 > n^2 2^n, \quad \forall n \geq 4$
\end{proposition}

\begin{proof}
	\hfill
	\begin{enumerate}[label=\arabic*.]
		\item Começamos com o caso $n = 4$: temos $(4!)^2 = 576 > 256 = 4^2 \cdot
			      2^4$, que é
		      uma afirmação verdadeira.
		\item Fixamos agora um $n \in \mathbb{N}: n\geq 4$ e assumimos que
		      \begin{align*}
			      (n!)^2     & > n^2 2^n
			      \quad \text{(Hipótese de Indução)}.
			      \intertext{Queremos mostrar que:}
			      [(n+1)!]^2 & > (n + 1)^2 2^{n + 1}
			      \quad \text{(Tese de Indução)}
		      \end{align*}
	\end{enumerate}
\end{proof}
\clearpage

\exercicio{3}

\begin{equation}
	\begin{cases}
		a_1 = \frac{1}{4} \\
		a_{n + 1} = (a^2_n + 1) \frac{a_n}{2}, n \in \mathbb{N}
	\end{cases}
\end{equation}

\begin{enumerate}[label=\arabic{section}.\arabic*.]
	\item
	      \begin{proposition}
		      $a_n \in ]0, 1[, \quad \forall n \in \mathbb{N}$
	      \end{proposition}
	\item
	      \begin{proposition}
		      A sucessão de $a_n$ é decrescente.
	      \end{proposition}
	\item
	      \begin{proposition}
		      A sucessão de $a_n$ é convergente.
	      \end{proposition}
	      \begin{align*}
		      \lim_{n} a_n =
	      \end{align*}
	\item
	      \begin{proposition}
		      Se $a_n$ é um subsucessão de uma sucessão limitada $b_n$, $b_n$ é
		      convergente.
	      \end{proposition}
\end{enumerate}

\exercicio{4}

\begin{equation}
	\begin{cases}
		u_1 = 2 \\
		u_{n + 1} = (n^2 + 1)u_n, \quad n \in \mathbb{N}
	\end{cases}
\end{equation}

\begin{enumerate}[label=\arabic{section}.\arabic*.]
	\item
	      \begin{proposition}
		      Para todo $n \in \mathbb{N}$, tem-se que $\frac{u_n}{n!}\geq 2$.
	      \end{proposition}
	\item Monotonia de $u_n$
	\item
	      \begin{proposition}
		      Existe um termo de $u_n$ que seja igual a 150.
	      \end{proposition}
	\item
	      \begin{proposition}
		      Existe limite da sucessão $u_n$.
	      \end{proposition}
\end{enumerate}

\clearpage

\exercicio{5}

\begin{enumerate}[label=\arabic{section}.\arabic*.]
	\item
	      \begin{equation}
		      0 < (n + 1)^\frac{1}{4}-n^\frac{1}{4}\leq n^{-\frac{3}{4}}, \quad
		      n \in \mathbb{N}
	      \end{equation}
	\item
	      \begin{align*}
		      \lim_{n} \left(\sqrt[4]{n + 1}-\sqrt[4]{n}\right)
	      \end{align*}
\end{enumerate}

\exercicio{6}

\begin{enumerate}[label=\arabic{section}.\arabic*.]
	\item
	      \begin{equation}
		      \sum_{n=1}^{\infty} \frac{n - 1}{\sqrt{n^3 + 2n^2 + 2}}
	      \end{equation}
	\item
	      \begin{equation}
		      \sum_{n=1}^{\infty} \left(\frac{2n}{3n -2\sqrt{n}}\right)^n
	      \end{equation}
	\item
	      \begin{equation}
		      \sum_{n=1}^{\infty} \left(\frac{2}{n^2}\right)^n n!
	      \end{equation}
	\item
	      \begin{equation}
		      \sum_{n=1}^{\infty} (-1)^n \frac{1}{\sqrt[3]{n}}
	      \end{equation}
\end{enumerate}

\clearpage

\exercicio{7}

\begin{equation}
	\alpha \in \mathbb{R}: \sum_{n=1}^{\infty}
	(-1)^n \frac{n^2}{(n + 1)^{\alpha + 1}}
\end{equation}

\begin{enumerate}[label=\arabic{section}.\arabic*.]
	\item
	      \begin{proposition}
		      A série é absolutamente convergente se, e só se, $\alpha > 2$.
	      \end{proposition}
	\item
	      \begin{proposition}
		      Para $\alpha \leq 1$, a série diverge.
	      \end{proposition}
\end{enumerate}

\exercicio{8}

\begin{proposition}\label{prop:af1-ex8-an-convergente}
	A série $\sum_{n=1}^{\infty} a_n$ é convergente.
\end{proposition}

\begin{proposition}\label{prop:af1-ex8-bn-divergente}
	A série $\sum_{n=1}^{\infty} b_n$ é divergente.
\end{proposition}


\begin{enumerate}[label=\arabic{section}.\arabic*.]
	\item
	      \begin{proposition}
		      A série $\sum_{n=1}^{\infty} \frac{a_n^2}{1 + b_n}$ é convergente.
	      \end{proposition}
	\item
	      \begin{proposition}
		      A série $\sum_{n=1}^{\infty} \left(a_n + \frac{1}{b_n}\right)$ é convergente.
	      \end{proposition}
\end{enumerate}

%\nocite{*}
%\printbibliography[title={Bibliografia},heading=bibintoc]
\end{document}
