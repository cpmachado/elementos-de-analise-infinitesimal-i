\exercicio{}

\begin{equation}\label{eq:efa-2}
	\sum_{n=0}^{\infty} \frac{1}{1 + \alpha^{2n}}
\end{equation}

\paragraph{} Seja $\alpha \in \mathbb{R}$, pretende-se determinar os valores
de $\alpha$ para qual a série, definida em \ref{eq:efa-2}, converge.

\begin{align*}
	\intertext{Começamos por observar que}
	\alpha^{2n}
	&= (\alpha^n)^2 \geq 0
	\intertext{Se $|\alpha| < 1$, tem-se que}
	\lim_{n \to \infty} 1 + \alpha^{2n}
	&= 1
	\iff \lim_{n \to \infty} \frac{1}{1 + a^{2n}} = 1.
	\intertext{Se $|\alpha| = 1$, tem-se que}
	\lim_{n \to \infty} 1 + \alpha^{2n}
	&= 2
	\iff \lim_{n \to \infty} \frac{1}{1 + a^{2n}} = \frac{1}{2}.
	\intertext{Se $|\alpha| > 1$, tem-se que}
	\lim_{n \to \infty} 1 + \alpha^{2n}
	&= +\infty
	\iff \lim_{n \to \infty} \frac{1}{1 + a^{2n}} = 0.
\end{align*}

\paragraph{}Pela alínea 1 do Teorema
4\footcite[pág. 579]{Santos2016}, a série só converge se o termo
geral convergir para 0. Logo, a série
$\sum_{n=0}^{\infty} \frac{1}{1+\alpha^{2n}}$ é convergente para
$\alpha \in ]-\infty, -1[ \cup ]1, +\infty[$.




