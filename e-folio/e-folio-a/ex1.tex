\exercicio{}

\begin{equation}
	\begin{cases}
		v_1 = 1 \\
		v_{n + 1} = \frac{v^2_n}{1 + v_n}, \quad n \in \mathbb{N}
	\end{cases}
\end{equation}

\begin{enumerate}[label=\arabic{section}.\arabic*.]
	\item
	      \begin{proposition}\label{prop:efa-1a}
		      Para qualquer $n \in \mathbb{N}$, tem-se que $0 < v_n\leq 1$.
	      \end{proposition}
	      \begin{proof}
		      \hfill\\
		      \begin{enumerate}[label=\arabic*.]
			      \item Começamos pelo caso $n = 1$, no qual tem-se que $v_1 = 1 \in]0, 1]$
			            , que é uma afirmação verdadeira.
			      \item Fixamos agora um $n \in \mathbb{N}$ e assumimos que
			            \begin{align*}
				                 & 0 < v_n \leq 1 \quad\quad \text{(Hipótese de Indução)}
				            \intertext{Queremos mostrar que:}
				                 & 0 < v_{n + 1} \leq 1 \quad \text{(Tese de Indução)}
				            \intertext{Procedemos ao Passo de indução, no qual
							com base na hipótese, tem-se que}
				                 & 0 < v_n \leq 1 \implies 0 < v_n^2 \leq 1               \\
				                 & 0 < v_n \leq 1 \iff \; 1 < 1 + v_n
				            \intertext{Que nos permite, deduzir que:}
				                 & 0 < v_n^2 \leq 1 < 1 + v_n                             \\
				            \iff & 0 < \frac{v_n^2}{1 + v_n} \leq \frac{1}{1 + v_n} < 1
				            \intertext{Dado que
					            $v_{n + 1} = \frac{v_n^2}{1 + v_n}$, podemos
					            concluir que está demonstrado que}
				                 & 0 < v_{n+1} \leq 1.
								 \intertext{Que nos permite concluir a nossa
									 demonstração da proposição, por indução,
									 no qual fica demonstrado que para $n \in
									 \mathbb{N}$, tem-se que $0 < v_n \leq
								 1$.}
			            \end{align*}
		      \end{enumerate}
	      \end{proof}
	\item
	      \begin{proposition}
		      A sucessão $(v_n)$ é monótona.
	      \end{proposition}
	      \begin{proof}
		      \; \\
		      De modo a verificar se a sucessão é monótona, vamos verificar se
		      $v_{n+1} - v_n$ é positivo ou negativo.
		      \begin{align*}
			      v_{n + 1} - v_n = \frac{v_n^2}{1 + v_n} - v_n
			      = \frac{v_n^2-v_n(1 + v_n)}{1 + v_n}
			      = -\frac{v_n}{1 + v_n}
		      \end{align*}
		      Com base na proposição \ref{prop:efa-1a}, do exercício anterior, tem-se que
		      \begin{align*}
			      0 < v_n \leq 1 \iff 1 < 1 + v_n \implies
			      v_{n + 1} - v_n = -\frac{v_n}{1 + v_n} < 0
		      \end{align*}
		      Que nos permite concluir que a sucessão $(v_n)$ é estritamente
		      decrescente, consequentemente monótona.\\
	      \end{proof}
	      \clearpage
	\item \hfill
	      \begin{enumerate}[label=\arabic{section}.\arabic{enumi}.\arabic*.]
			  \item \; \\
				  Pretende-se estudar a natureza da série, definida por:
		            \begin{equation}\label{eq:efa-1-3-1}
			            \sum_{n=1}^{\infty} v_n
		            \end{equation}
				  \begin{align*}
					  \intertext{Pela Proposição \ref{prop:efa-1a}, do
					  exercício 1.1, tem-se que}
					  0 < v_n \leq 1 \iff &1 < 1 + v_n \\
					  \iff &
					  0 < \frac{v_n}{1+v_n} < 1.
					  \intertext{Donde se tem que:}
					  \lim_{n \to \infty} \left|\frac{v_{n + 1}}{v_n}\right|
					  &=
					  \lim_{n \to \infty}
					  \left|\frac{\frac{v_n^2}{1+v_n}}{v_n}\right|\\
					  &=
					  \lim_{n \to \infty}
					  \left|\frac{v_n^2}{v_n(1+v_n)}\right|\\
					  &=
					  \lim_{n \to \infty}
					  \left|\frac{v_n}{1+v_n}\right| < 1\\
					  \intertext{Deste modo concluímos, pelo Critério
						  D'Alembert(Teorema 7, alínea 1), que a série é
					  absolutamente convergente.}
				  \end{align*}
			\item \; \\
				  Pretende-se estuda a natureza da série, definida por:
				  \begin{equation}\label{eq:efa-1-3-2}
			            \sum_{n=1}^{\infty} \frac{v_n}{1 + v_n}
		          \end{equation}
				  Vamos começar por verificar se os termos gerais
				  da série $\sum_{n=1}^{\infty} \frac{v_n}{1 + v_n}$  e da
				  série $\sum_{n=1}^{\infty} v_n$, do exercício anterior, são
				  assimptoticamente iguais
				  \footcite[pág. 384, Definição 7]{Santos2016}:
				  \begin{align*}
					  \lim_{n \to \infty} \frac{\frac{v_n}{1+vn}}{v_n} =
					  \lim_{n \to \infty} \frac{1}{1 + v_n} = 1 \implies
					  \frac{v_n}{1 + v_n} \sim v_n
				  \end{align*}
				  Pelo Critério da
				  comparação(Teorema 6, alínea 1)\footcite[pág.
				  595]{Santos2016}, dado que
				  $\sum_{n=1}^{\infty} \frac{v_n}{1 + v_n}$ e
				  $\sum_{n=1}^{\infty} v_n$ são ambas séries de termos
				  não negativos, e assimptoticamente iguais podemos concluir 
				  que as séries têm a mesma natureza. Logo a série
				  $\sum_{n=1}^{\infty} \frac{v_n}{1 + v_n}$ é absolutamente
				  convergente.
	      \end{enumerate}
\end{enumerate}
