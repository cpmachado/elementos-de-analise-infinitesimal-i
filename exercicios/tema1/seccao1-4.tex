\seccao{1.4: Somatórios e progressões geométricas}

\exercicio{1}

\begin{align*}
	\sum^2_{k = 0} (3k^2 -1)
\end{align*}


\exercicio{3}

\begin{align*}
	\sum_{n=2}^{4} \frac{1}{n^2 - 1}
\end{align*}


\exercicio{6}

\begin{align*}
	7 + 9 + 11 + 13 + \ldots + 37 = 
\end{align*}

\exercicio{7}

\begin{align*}
	\frac{1}{4} + \frac{1}{8} + \frac{1}{16} + \frac{1}{32} + \ldots +
	\frac{1}{512}
\end{align*}


\exercicio{9}

\begin{enumerate}
	\item
		\begin{proposition}
			Para qualquer $n \in \mathbb{N}$, tem-se que
			$\sum_{k=1}^{n} k = \frac{n(n+1)}{2}$.
		\end{proposition}
		\begin{proof}
			Algo
		\end{proof}
\end{enumerate}

\exercicio{10}

\begin{align*}
	\sum_{k=1}^{n} (2k+3)
\end{align*}

