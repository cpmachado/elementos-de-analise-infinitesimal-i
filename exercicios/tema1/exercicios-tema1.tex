\documentclass[11pt, a4paper]{article}
\usepackage[utf8]{inputenc}
\usepackage[portuguese]{babel}
\usepackage[T1]{fontenc}
\usepackage[margin=2cm]{geometry}
\usepackage{amsmath}
\usepackage{amsfonts}
\usepackage{amsthm}
\usepackage{mathtools}
\usepackage{enumitem}
\usepackage{csquotes}
\usepackage[citestyle=authortitle]{biblatex}
\usepackage[hidelinks]{hyperref}

\addbibresource{bibliografia.bib}

\title{
	Elementos de Análise Infinitesimal I\\
	Exercícios do Tema 1: Números e Sucessões\\
	Proposta de Resolução
}

\author{
	Carlos Pinto Machado
	<\href{mailto:2200909@estudante.uab.pt}{2200909@estudante.uab.pt}>
}

\hypersetup{
	pdfsubject = {Elementos de Análise Infinitesimal I},
	pdftitle = {Exercícios do Tema 1: Números e Sucessões(Proposta de Resolução)},
	pdfauthor = {Carlos Pinto Machado(2200909)},
	pdfcreator = {},
	pdfproducer = {}
}

\newcommand{\seccao}[1]{
	\chapter*{Secção #1}
	\phantomsection
	\addcontentsline{toc}{chapter}{Secção #1}
}

\newcommand{\exercicio}[1]{
	\section*{#1.}
	\phantomsection
	\addcontentsline{toc}{section}{#1.}
	\setcounter{exercicio}{#1}
	\setcounter{proposition}{0}
}

\newcounter{exercicio}

\newtheorem{proposition}{Proposição}[exercicio]


\begin{document}
\maketitle
\tableofcontents

\clearpage

\section*{Introdução}
\phantomsection
\addcontentsline{toc}{section}{Introdução}

\paragraph{} Todos os exercícios referem se à bibliografia obrigatória da
cadeira\parencite{Santos2016}.

\seccao{1.3}

\exercicio{8}

\begin{equation}\label{eq:1.3-8}
	x_n =
	\begin{cases}
		1, &n = 0\\
		x_n(x_n + 1), &n \in \mathbb{N}_0
	\end{cases}
\end{equation}

\begin{enumerate}[label=(\alph*)]
	\item
		\begin{proposition}
			Para qualquer $n \in \mathbb{N}_0$, tem-se que $x_n > 0$, onde
			$x_n$ é a sucessão definida na equação \ref{eq:1.3-8}.
		\end{proposition}
		\begin{proof}
			\hfill
			\begin{enumerate}[label=\arabic*.]
				\item Começamos com o caso $n = 0$: temos $x_0 = 1 > 0$, que é
					uma afirmação verdadeira.
				\item Fixamos agora um $n \in \mathbb{N}_0$ e assumimos que
					\begin{align*}
						&\text{Hipótese}:\quad  x_n > 0.
						\intertext{Queremos mostrar que:}
						&\text{Tese}: \quad\quad  x_{n + 1} > 0.
						\intertext{Comecemos por considerar a Hipótese,}
						&x_n > 0 \iff x_n + 1 > 1
						\intertext{Pela Hipótese, podemos
							multiplicar ambos os termos da inequação sem
						inverter o sinal, pelo que se tem}
						&x_{n + 1} = x_n(x_n + 1) > x_n > 0
					\end{align*}
			\end{enumerate}
		\end{proof}
	\item
		\begin{proposition}
			Para qualquer $n \in \mathbb{N}$, tem-se que $x_n$ é par, onde
			$x_n$ é a sucessão definida na equação \ref{eq:1.3-8}.
		\end{proposition}
\end{enumerate}


\exercicio{9}

\exercicio{11}

\exercicio{14}

\exercicio{17}

%\clearpage
%\seccao{1.4: Somatórios e progressões geométricas}

\exercicio{1}

\begin{align*}
	\sum^2_{k = 0} (3k^2 -1)
\end{align*}


\exercicio{3}

\begin{align*}
	\sum_{n=2}^{4} \frac{1}{n^2 - 1}
\end{align*}


\exercicio{6}

\begin{align*}
	7 + 9 + 11 + 13 + \ldots + 37 = 
\end{align*}

\exercicio{7}

\begin{align*}
	\frac{1}{4} + \frac{1}{8} + \frac{1}{16} + \frac{1}{32} + \ldots +
	\frac{1}{512}
\end{align*}


\exercicio{9}

\begin{enumerate}
	\item
		\begin{proposition}
			Para qualquer $n \in \mathbb{N}$, tem-se que
			$\sum_{k=1}^{n} k = \frac{n(n+1)}{2}$.
		\end{proposition}
		\begin{proof}
			Algo
		\end{proof}
\end{enumerate}

\exercicio{10}

\begin{align*}
	\sum_{k=1}^{n} (2k+3)
\end{align*}


%\seccao{2.2}

\exercicio{5}

\exercicio{6}

\exercicio{7}

\exercicio{10}

\exercicio{12}

\exercicio{17}

\exercicio{18}

%\seccao{2.5: Limites de sucessão}

\exercicio{1}

\exercicio{2}

\exercicio{15}

\begin{align*}
	\lim_{n \to 2} \frac{2n - 1}{n + 1}
\end{align*}

\exercicio{21}


\clearpage

\printbibliography[title={Bibliografia},heading=bibintoc]

\end{document}
