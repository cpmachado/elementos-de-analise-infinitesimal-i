\documentclass[11pt, a4paper]{article}
\usepackage[utf8]{inputenc}
\usepackage[portuguese]{babel}
\usepackage[T1]{fontenc}
\usepackage[margin=2cm]{geometry}
\usepackage{amsmath}
\usepackage{amsfonts}
\usepackage{amsthm}
\usepackage{mathtools}
\usepackage{enumitem}
\usepackage{csquotes}
\usepackage[citestyle=authoryear]{biblatex}
\usepackage[hidelinks]{hyperref}

\addbibresource{bibliografia.bib}

\title{
	Elementos de Análise Infinitesimal I\\
	Exercícios do Tema 1: Números e Sucessões\\
	Proposta de Resolução
}

\author{
	Carlos Pinto Machado
	<\href{mailto:2200909@estudante.uab.pt}{2200909@estudante.uab.pt}>
}

\hypersetup{
	pdfsubject = {Elementos de Análise Infinitesimal I},
	pdftitle = {Exercícios do Tema 1: Números e Sucessões(Proposta de Resolução)},
	pdfauthor = {Carlos Pinto Machado(2200909)},
	pdfcreator = {},
	pdfproducer = {}
}

\newtheorem{proposicao}{Proposição}

\newcommand{\seccao}[1]{
	\chapter*{Secção #1}
	\phantomsection
	\addcontentsline{toc}{chapter}{Secção #1}
}

\newcommand{\exercicio}[1]{
	\section*{#1.}
	\phantomsection
	\addcontentsline{toc}{section}{#1.}
	\setcounter{exercicio}{#1}
	\setcounter{proposition}{0}
}

\newcounter{exercicio}

\newtheorem{proposition}{Proposição}[exercicio]


\begin{document}
\maketitle

\seccao{1.3: Sucessões e indução}

\exercicio{8}

\begin{equation}\label{eq:1.3-8}
	x_n =
	\begin{cases}
		1, &n = 0\\
		x_{n - 1}(x_{n - 1} + 1), &n \in \mathbb{N}_0
	\end{cases}
\end{equation}

\begin{enumerate}[label=(\alph*)]
	\item
		\begin{proposition}
			Para qualquer $n \in \mathbb{N}_0$, tem-se que $x_n > 0$, onde
			$x_n$ é a sucessão definida na equação \ref{eq:1.3-8}.
		\end{proposition}
		\begin{proof}
			\hfill
			\begin{enumerate}[label=\arabic*.]
				\item Começamos com o caso $n = 0$: temos $x_0 = 1 > 0$, que é
					uma afirmação verdadeira.
				\item Fixamos agora um $n \in \mathbb{N}_0$ e assumimos que
					\begin{align*}
						x_n &> 0 \quad \text{(Hipótese de Indução)}.
						\intertext{Queremos mostrar que:}
						x_{n + 1} &> 0 \quad \text{(Tese de Indução)}
						\intertext{Procedemos ao Passo de Indução, ao começar
						com a inequação da Hipótese assumida:}
						x_n &> 0 \\
						\iff x_n + 1 &> 1 \\
						\iff x_n(x_n + 1) &> x_n > 0
						\quad \text{(Pela Hip. de Indução)}
						\intertext{Dado que $x_{n + 1} = x_n(x_n + 1)$,
						podemos concluir que}
						x_{n + 1} = x_n(x_n + 1) &> x_n > 0
					\end{align*}
					Deste modo provamos que para qualquer $n\in\mathbb{N}_0$,
					tem-se que $x_n > 0$.
			\end{enumerate}
		\end{proof}
		\clearpage
	\item
		\begin{proposition}
			Para qualquer $n \in \mathbb{N}$, tem-se que $x_n$ é par, onde
			$x_n$ é a sucessão definida na equação \ref{eq:1.3-8}.
		\end{proposition}
		\begin{proof}
			\hfill
			\begin{enumerate}[label=\arabic*.]
				\item Começamos com o caso $n = 1$: temos $x_1 = 1(1 + 1) = 2$
					é par, que é uma afirmação verdadeira.
				\item Fixamos agora um $n \in \mathbb{N}$ e assumimos que
					\begin{align*}
						x_n &\text{ é par} \quad \text{(Hipótese de Indução)}.
						\intertext{Queremos mostrar que:}
						x_{n + 1} &\text{ é par} \quad \text{(Tese de Indução)}
					\end{align*}
					Começamos por considerar a definição do termo $x_{n + 1}$,
					que é $x_n(x_n + 1)$. Pela Hipótese de Indução, temos que
					$x_n \in \mathbb{N}$ e que é par, logo $x_{n + 1}$ vai ser
					o produto de um número par pelo seu sucessor, que
					forçosamente leva a que o produto tenha 2 como divisor e
					seja par.\\
					Deste modo provamos que para qualquer $n\in\mathbb{N}$,
					tem-se que $x_n$ é par.
			\end{enumerate}
		\end{proof}
\end{enumerate}

\clearpage

\exercicio{9}

\begin{proposition} \label{prop:1.3-9}
	Para $n$ par, $(-1)^n = 1$, e para $n$ ímpar, $(-1)^n = -1$.
\end{proposition}

\begin{proof}
	\hfill \\
	De modo a provar a proposição \ref{prop:1.3-9}, temos de provar
	separadamente se: $n$ é par, então $(-1)^n = 1$;
	e se $n$ é ímpar, então $(-1)^n = -1$. \\
	Vamos começar por provar se $n$ é par, então $(-1)^n = 1$. Comecemos por
	observar que para garantir que $n$ é par, vamos considerar antes a
	proposição $(-1)^{2n} = 1$.\\
	\begin{enumerate}[label=\arabic*.]
		\item Começamos com o caso $n = 1$: temos $(-1)^2 = 1$,
			que é uma afirmação verdadeira.
		\item Fixamos agora um $n \in \mathbb{N}$ e assumimos que
			\begin{align*}
				(-1)^{2n} &=1 \quad \text{(Hipótese de Indução)}.
				\intertext{Queremos mostrar que:}
				(-1)^{2(n+1)}&=1 \quad \text{(Tese de Indução)}
				\intertext{Começamos por considerar a Tese de Indução:}
				(-1)^{2(n+1)}
				&= (-1)^{2n + 2} \\
				&= (-1)^{2n} \cdot (-1)^2\\
				&= 1 \cdot 1 \quad\text{(Pela Hipótese de Indução)}\\
				&= 1
			\end{align*}
	\end{enumerate}
	Deste modo fica provado que se $n$ é par, $(-1)^n = 1$.\\
	Agora vamos demonstrar que se $n$ é ímpar, $(-1)^n = -1$. De forma
	semelhante, vamos considerar antes a proposição $(-1)^{2n - 1} = -1$, de
	modo a garantir que $n$ é ímpar.
	\begin{enumerate}[label=\arabic*.]
		\item Começamos com o caso $n = 1$: temos $(-1)^1 = -1$,
			que é uma afirmação verdadeira.
		\item Fixamos agora um $n \in \mathbb{N}$ e assumimos que
			\begin{align*}
				(-1)^{2n - 1} &= -1 \quad \text{(Hipótese de Indução)}.
				\intertext{Queremos mostrar que:}
				(-1)^{2(n+1) - 1}&=-1 \quad \text{(Tese de Indução)}
				\intertext{Começamos por considerar a Tese de Indução:}
				(-1)^{2(n+1) - 1}
				&= (-1)^{2n + 2 - 1} \\
				&= (-1)^{2n - 1} \cdot (-1)^2\\
				&= (-1) \cdot 1 \quad\text{(Pela Hipótese de Indução)}\\
				&= -1
			\end{align*}
	\end{enumerate}
	Deste modo fica provado que se $n$ é ímpar, $(-1)^n = -1$.\\
	Provados ambos casos, a proposição \ref{prop:1.3-9} fica provada.\\
\end{proof}


\clearpage
\exercicio{11}

\begin{proposition}
	$2n - 3 < 2^{n - 2}$, para todo o inteiro $n\geq 5$.
\end{proposition}

\exercicio{14}

\begin{equation}\label{eq:1.3-14}
	x_n =
	\begin{cases}
		1, &n = 1\\
		\frac{2x_{n - 1} + 3}{4}, &n \geq 1
	\end{cases}
\end{equation}

\begin{proposition}
	Para qualquer $n \in \mathbb{N}$, tem-se que $x_n < \frac{3}{2}$, onde
	$x_n$ é a sucessão definida na equação \ref{eq:1.3-14}.
\end{proposition}

\exercicio{17}

\begin{proposition}
	Para qualquer $n \in \mathbb{N}_0$, tem-se que $a^nb^n = (ab)^n$.
\end{proposition}

%\seccao{1.4: Somatórios e progressões geométricas}

\exercicio{1}

\begin{align*}
	\sum^2_{k = 0} (3k^2 -1)
\end{align*}


\exercicio{3}

\begin{align*}
	\sum_{n=2}^{4} \frac{1}{n^2 - 1}
\end{align*}


\exercicio{6}

\begin{align*}
	7 + 9 + 11 + 13 + \ldots + 37 = 
\end{align*}

\exercicio{7}

\begin{align*}
	\frac{1}{4} + \frac{1}{8} + \frac{1}{16} + \frac{1}{32} + \ldots +
	\frac{1}{512}
\end{align*}


\exercicio{9}

\begin{enumerate}
	\item
		\begin{proposition}
			Para qualquer $n \in \mathbb{N}$, tem-se que
			$\sum_{k=1}^{n} k = \frac{n(n+1)}{2}$.
		\end{proposition}
		\begin{proof}
			Algo
		\end{proof}
\end{enumerate}

\exercicio{10}

\begin{align*}
	\sum_{k=1}^{n} (2k+3)
\end{align*}


%\seccao{2.2: Supremo}

\exercicio{5}

\begin{align*}
	[0, 1[
\end{align*}

\exercicio{6}

\begin{align*}
	]-1,0] \cup \{1\}
\end{align*}

\exercicio{7}

\begin{align*}
	[-2,2]\backslash\{1\}
\end{align*}

\exercicio{10}

\begin{align*}
	\mathbb{Z} \cup \left]-1, \frac{3}{2}\right[
\end{align*}

\exercicio{12}

\begin{align*}
	[0,1]\cup\{2\}\cup]3,4[
\end{align*}

\exercicio{17}

\begin{proposition}
	Se $\text{sup} \; A \geq \text{inf}\;B$, então $A$ e $B$ são disjuntos.
\end{proposition}

\exercicio{18}
\begin{proposition}
	Se $\text{sup} \; A \geq \text{inf}\;B$, então $A$ e $B$ não podem ser
	disjuntos.
\end{proposition}

%\seccao{2.5: Limites de sucessão}

\exercicio{1}

\exercicio{2}

\exercicio{15}

\begin{align*}
	\lim_{n \to 2} \frac{2n - 1}{n + 1}
\end{align*}

\exercicio{21}


\clearpage

\nocite{*}
\printbibliography[title={Bibliografia},heading=bibintoc]

\end{document}
