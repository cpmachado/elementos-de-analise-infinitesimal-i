\seccao{1.3}

\exercicio{8}

\begin{equation}\label{eq:1.3-8}
	x_n =
	\begin{cases}
		1, &n = 0\\
		x_n(x_n + 1), &n \in \mathbb{N}_0
	\end{cases}
\end{equation}

\begin{enumerate}[label=(\alph*)]
	\item
		\begin{proposition}
			Para qualquer $n \in \mathbb{N}_0$, tem-se que $x_n > 0$, onde
			$x_n$ é a sucessão definida na equação \ref{eq:1.3-8}.
		\end{proposition}
		\begin{proof}
			\hfill
			\begin{enumerate}[label=\arabic*.]
				\item Começamos com o caso $n = 0$: temos $x_0 = 1 > 0$, que é
					uma afirmação verdadeira.
				\item Fixamos agora um $n \in \mathbb{N}_0$ e assumimos que
					\begin{align*}
						&\text{Hipótese}:\quad  x_n > 0.
						\intertext{Queremos mostrar que:}
						&\text{Tese}: \quad\quad  x_{n + 1} > 0.
						\intertext{Comecemos por considerar a Hipótese,}
						&x_n > 0 \iff x_n + 1 > 1
						\intertext{Pela Hipótese, podemos
							multiplicar ambos os termos da inequação sem
						inverter o sinal, pelo que se tem}
						&x_{n + 1} = x_n(x_n + 1) > x_n > 0
					\end{align*}
			\end{enumerate}
		\end{proof}
	\item
		\begin{proposition}
			Para qualquer $n \in \mathbb{N}$, tem-se que $x_n$ é par, onde
			$x_n$ é a sucessão definida na equação \ref{eq:1.3-8}.
		\end{proposition}
\end{enumerate}


\exercicio{9}

\exercicio{11}

\exercicio{14}

\exercicio{17}
