\documentclass[11pt, a4paper]{report}
\usepackage[utf8]{inputenc}
\usepackage[portuguese]{babel}
\usepackage[T1]{fontenc}
\usepackage[margin=2cm]{geometry}
\usepackage{amsmath}
\usepackage{amsfonts}
\usepackage{amsthm}
\usepackage{mathtools}
\usepackage{enumitem}
\usepackage{csquotes}
\usepackage[citestyle=authortitle]{biblatex}
\usepackage[hidelinks]{hyperref}

\addbibresource{bibliografia.bib}

\title{
	Elementos de Análise Infinitesimal I\\
	Exercícios do Tema 2: Séries\\
	Proposta de Resolução
}

\author{
	Carlos Pinto Machado
	<\href{mailto:2200909@estudante.uab.pt}{2200909@estudante.uab.pt}>
}

\hypersetup{
	pdfsubject = {Elementos de Análise Infinitesimal I},
	pdftitle = {Exercícios do Tema 2: Séries(Proposta de Resolução)},
	pdfauthor = {Carlos Pinto Machado(2200909)},
	pdfcreator = {},
	pdfproducer = {}
}

\newcommand{\seccao}[1]{
	\chapter*{Secção #1}
	\phantomsection
	\addcontentsline{toc}{chapter}{Secção #1}
}

\newcommand{\exercicio}[1]{
	\section*{#1.}
	\phantomsection
	\addcontentsline{toc}{section}{#1.}
	\setcounter{exercicio}{#1}
	\setcounter{proposition}{0}
}

\newcounter{exercicio}

\newtheorem{proposition}{Proposição}[exercicio]


\begin{document}
\maketitle
\tableofcontents

\clearpage

\chapter*{Introdução}
\phantomsection
\addcontentsline{toc}{chapter}{Introdução}

\paragraph{} Todos os exercícios referem se à bibliografia obrigatória da
cadeira\parencite{Santos2016}.

\seccao{9.2 - Séries: definições e primeiras propriedades}

\exercicio{9}

\exercicio{10}

\exercicio{15}

\exercicio{26}

\exercicio{27}

\exercicio{28}

\exercicio{31}

\exercicio{44}

\exercicio{46}

\seccao{9.3 - Séries de termos não negativos}

\exercicio{1}

\exercicio{3}

\exercicio{6}

\exercicio{15}

\exercicio{29}

\exercicio{30}

\exercicio{32}

\seccao{9.4 - Séries alternadas e convergência absoluta}

\exercicio{1}

\exercicio{10}

\exercicio{11}

\exercicio{15}

\exercicio{16}

\exercicio{18}

\exercicio{21}

\exercicio{24}

\exercicio{25}

\exercicio{26}


\clearpage

\printbibliography[title={Bibliografia},heading=bibintoc]

\end{document}
